\documentclass[UTF8]{ctexart}
%%%%%%%%%%%%%%%%%%%%%%%%%%%%%%%%%%%%%%%%%%%%%%%%%%%
%协议:MIT协议(http://opensource.org/licenses/MIT)
%%%%%%%%%%%%%%%%%%%%%%%%%%%%%%%%%%%%%%%%%%%%%%%%%%%

%对于包的使用

%超链接
\usepackage[colorlinks,linkcolor=black]{hyperref}
%板式
\usepackage{geometry}
%图片
\usepackage{graphicx}
%图表标题
\usepackage{bicaption}
%子图
\usepackage{subcaption}
%页眉页脚控制
\usepackage{fancyhdr}
%非汉字字体设置
\usepackage{fontspec}
%控制参考文献格式
\usepackage[square,super, comma, sort&compress, numbers]{natbib}
%计数器控制
\usepackage{amsmath}
%插入代码
%\usepackage{listings}
%设置列表格式
%\usepackage{enumitem}
%%%%%%%%%%%%%%%%%%%%%%%%%%%%%%%%%%%%%%%%%%%%%%%%%%%

%论文一律编排并打印在A4幅面白纸上
%上边距为25mm,下边距为25mm,左边距为30mm,右边距为20mm
\geometry{a4paper,left=30mm,right=20mm,top=25mm,bottom=25mm}

%论文行间距为1.5倍(可能需要用\selectfont)
\linespread{1.5}

%页眉页脚开启
\pagestyle{fancy}
%页眉字号字体为小五号宋体,内容为“北京林业大学本科毕业论文(设计)”
\chead{\zihao{-5}\CJKfamily{zhsong}北京林业大学本科毕业论文(设计) } 
%页眉只有中间有内容,左右为空白
\lhead{} 
\rhead{}

%页眉下边的横线宽度:事实上论文规范并没有对此做出规定,但是我们选择遵循常理去除横线。
\renewcommand{\headrulewidth}{0pt}

%论文页码的字号字体为小五号Times New Roman体
%从绪论部分开始,至附录,用阿拉伯数字连续编排
%页码位于页面底端居中
\renewcommand{\footnotesize}{\zihao{-5}}  

%论文中出现的所有数字和字母如无特殊要求都用Times New Roman体。

%设置正文非汉字的字体为Times New Roman
\setmainfont{Times New Roman}
%设置正文非汉字的粗体字体为Times New Roman
\setsansfont{Times New Roman}

%设置英文图为Fig,表格为Tab
\DeclareCaptionOption{english}[]{
\renewcommand\figurename{Fig.}
\renewcommand\tablename{Tab}}
\captionsetup[bi-second]{english}

%一定要在这里声明space,直接声明不好使,小五号
\captionsetup{font={footnotesize,bf},labelsep=space}

%设置图片编号随着章节自动清零
\numberwithin{figure}{section}
%设置公式编号随着章节自动清零
\numberwithin{equation}{section}

%定义图的编号为章节号码.序号
\renewcommand{\thefigure}{\thesection.\arabic{figure}}
%定义公式的编号为章节号码.序号
\renewcommand{\theequation}{\thesection.\arabic{equation}}

%改变figure为图
\renewcommand{\figureautorefname}{图}

%改变equation为式
\renewcommand{\equationautorefname}{式}
\renewcommand{\subsectionautorefname}{节}
\renewcommand{\subsectionautorefname}{章}

%章标题:四号宋体加粗
%section另起一页,宋体四号加粗居中
\CTEXsetup[format={\pagebreak\centering\zihao{4}\CJKfamily{zhsong}\textbf}]{section}

%节标题:小四号宋体加粗
%subsection宋体小四号加粗
\CTEXsetup[format={\zihao{-4}\CJKfamily{zhsong}\textbf}]{subsection}

%代码报lsting格式
%\lstset{numbers=none,
%  numberstyle=\scriptsize,
%  frame=lines,
%  flexiblecolumns=false,
%  language=Python,
%  basicstyle=\ttfamily\small,
%  breaklines=true,
%  extendedchars=true,
%  escapechar=\%,
%  texcl=true,
%  showstringspaces=true,
%  keywordstyle=\bfseries,
%  tabsize=4}

%%%%
%以下内容理工农医类专业请直接注释掉
%设置section计数器为汉字
\CTEXsetup[number={\chinese{section}}]{section}
\CTEXsetup[name={(,)}]{subsection}
\CTEXsetup[number={\chinese{subsection}}]{subsection}


%%%%%%%%%%%%%%%%%%%%%%%%%%%%%%%%%%%%%%%%%%%%%%%%%%%
\title{北京林业大学本科毕业论文\\模板}
\author{作者}
%\date{\today}

%国标中文文献格式GBT7714-2005NLang-UTF8
\bibliographystyle{GBT7714-2005NLang-UTF8}
%%%%%%%%%%%%%%%%%%%%%%%%%%%%%%%%%%%%%%%%%%%%%%%%%%%

\begin{document}

%论文摘要和目录用罗马数字单独编页码。
\pagenumbering{Roman}

\begin{titlepage}
%中文标题
\begin{center}

%论文题目用小三号黑体(中文)或者Times New Roman体(英文)
%分行用双反斜线表示
\zihao{-3}\CJKfamily{zhhei}\textbf{北京林业大学本科毕业论文\\模板}\\

%作者及指导教师用小四号宋体
%中文作者名,空格用单反斜线或者回车行表示
\zihao{-4}\CJKfamily{zhsong}班级\ 作者\\
\zihao{-4}\CJKfamily{zhsong}指导老师\ 教师\\

\end{center}

%%%%%%%%%%%%%%%%%%%%%%%%%%%%%%%%%%%%%%%%%%%%

%“摘要”字样用四号宋体加粗居中
\begin{center}
\zihao{4}\CJKfamily{zhsong}\textbf{摘要}
\end{center}

%摘要正文用楷体五号
\zihao{5}\CJKfamily{zhkai}根据《北京林业大学论文撰写规范及模板》, 本科生毕业论文必须符合相关规定。 一方面, 这些规定不尽不实, 存在着相当多的错漏; 另一方面, 即将毕业的学生实在很难静下来调整论文格式。 因此, 我们一起构建了本模板。\\\\

%“关键词”字样后带冒号,中文关键词为五号宋体加粗
\zihao{5}\CJKfamily{zhsong}\textbf{关键词:毕业论文, 论文模板, 北京林业大学, 本科生}
\pagebreak

%%%%%%%%%%%%%%%%%%%%%%%%%%%%%%%%%%%%%%%%%%%%%%%%%%%

%英文题目用小三号Times New Roman体,其他要求同中文
\begin{center}
\zihao{-3}\textbf{An Bachelor Thesis Template of Beijing Forestry University }\\
\zihao{-4}Class\ Author\\
\zihao{-4}Supervisor\ Your Teacher\\
\zihao{4}\textbf{Abstract}
\end{center}

%英文摘要正文
\zihao{5}In Beijing Forestry University, bachelor thesis have to meet some requirements. But relevant regulation is not consistency and complete. And it's hard for undergraduate to calm down and proofread their newly completed work. So, we construct this template in order to reduce their cost. \\\\

%英文关键词为五号Times New Roman体加粗
\zihao{5}\textbf{Key Words:Thesis, Template, Beijing Forestry University, Bachelor}

%%%%%%%%%%%%%%%%%%%%%%%%%%%%%%%%%%%%%%%%%%%%%%%%%%%
\end{titlepage}

\pagebreak

%自动生成目录页
\tableofcontents

\pagebreak

%自正文开始,使用阿拉伯数字且重新开始页码
%论文页码的字号字体为小五号Times New Roman体,页码位于页面底端居中;
\pagenumbering{arabic}

\section{快速开始}
\cite{刘海洋2013LATEX}

\section{作者名单及特别鸣谢}
王政

鸣谢马起园先生, 马先生作为ApTex的作者在2015年本模板的早期编写阶段给出了宝贵建议, 给予了极大的帮助。

%用biblatex来管理参考文献
\bibliography{database}
\end{document}
